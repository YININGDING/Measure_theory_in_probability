% !TEX root = ../mat999.tex
\newpage
\section{Convergence}
\subsection{$L^p$ spaces (Lebesgue spaces)}
The spaces of all integrable functions is defined by $L^1$ or $L_1 (\Omega, \F, \mu)$, i.e, $f \in \ls{1}$ iff $f$ is measurable and $\int |f| d\mu < \infty$ \\
$\ls{1}$ is a vector space as 
\begin{itemize}
    \item $|\alpha f + \beta g| \leq |\alpha||f| + |\beta||g|$
    \item Integral is a linear function on $\ls{1}$ by Theorem 6.4 (1, 2)
    \begin{itemize}
        \item $f+g$ is integrable and $\int f+g d\mu = \int f d\mu +\int g d\mu$
        \item If $c\in\R$ then $\int cf d\mu = c\int f d\mu$
    \end{itemize}
\end{itemize}
\begin{dfn}
For $p \geq 1$, $\ls{p}$ or $\ls{p}(\Omega, \F, \mu)$ is the space of all measurable functions $f$ s.t. 
\begin{equation*}
    \int |f|^p d\mu < \infty \qquad (\text{i.e. } |f|^p \in \ls{1})
\end{equation*}
\end{dfn}
The space $\ls{p}$ is equipped with p-norm given by $\norm{f}_p = \left( \into \abs{f}^p d\mu \right)^{\frac{1}{p}}$ \\
$\norm{\cdot}_p$ satisfies:
\begin{enumerate}
    \item $\norm{f}_p = 0 \Longleftrightarrow f = 0$ a.e.
    \item $\norm{cf}_p = \abs{c}\norm{f}_p, \quad c\in \R$
    \item $\norm{f+g}_p \leq \norm{f}_p +\norm{g}_p$ (Minkowski's inequality) \\
    \textbf{Proof of Minkowski's inequality is at section inequality}
    \item $\norm{f+g}_p^p \leq 2^{p-1}(\norm{f}_p^p +\norm{g}_p^p) $
\end{enumerate}
\pf 4 \\
Notice $h(t) = t^p \quad (p \geq 1)$ is convex for $t \geq 0$ For $x,y\geq 0$ we have 
\begin{align*}
    h(\frac{1}{2}x + \frac{1}{2}y) &\leq \frac{1}{2}(h(x)+h(y)) \\
    \frac{(x+y)^p}{2^p} &\leq  \frac{1}{2}(x^p + y^p) \\
    \text{WLOG, we assume } f,g \geq 0 &\text{ This will make it larger on the LHS but RHS stays the same} \\
    \frac{(\abs{f(\omega)+g(\omega)})^p}{2^p} &\leq  \frac{1}{2}(\abs{f(\omega)}^p + \abs{g(\omega)}^p) \qquad \forall \omega \in \Omega\\
    \text{Now integrate over $\Omega$ w.r.t } \mu \\
    \norm{f+g}_p^p &\leq 2^{p-1}(\norm{f}_p^p +\norm{g}_p^p)
\end{align*}
\qed
\newpage
$(V, \norm{\cdot})$ is normed vector space over a field $(\R)$ if
\begin{enumerate}
    \item $V$ is a vector space over field
    \item $\norm{\cdot}$ is a norm on $V$ over the field
    \begin{enumerate}
        \item $\norm{v}\geq 0$ and $\norm{v} \Longleftrightarrow V = 0_v \forall v\in V$
        \item $\norm{\alpha v} = \abs{\alpha}\norm{v}, \quad \forall\alpha\in \R, v\in V$
        \item $\norm{u+v} \leq \norm{u}+\norm{v}, \quad \forall u,v\in V$
    \end{enumerate}
\end{enumerate}
We denote by $\ls{p}(\Omega, \F, \mu)$ the space:
\begin{equation*}
    \ls{p} = \{f:\Omega \mapsto \R| \text{measurable and } \norm{f}_p = \left( \into \abs{f}^p d\mu \right)^{\frac{1}{p}} < \infty\} 
\end{equation*}quotient the space by equivalence relation $\sim$, that is:
\begin{equation*}
    f,g\in \ls{p}, f\sim g \Longleftrightarrow f-g = 0 \text{ a.e. } \mu
\end{equation*}Because without this, $\norm{f}_p = 0 \notimplies f = 0$, say for Lebesgue measure on $[0,1],$ 1 for rational and 0 for irrational.
\begin{thm}[Banach space]
$\ls{p}, p \geq 1$ is complete normed vector space (Banach space) i.e. every Cauchy sequence $\{f_n\}\subset \ls{p}$ has a limit in $\ls{p}$
\end{thm}
\textbf{Cauchy sequence:}$\forall \epsilon > 0, \exists N >0 , \forall n,m \geq N, \norm{f_n - f_m} \leq \epsilon$  \\
$\ls{2}$ is even more special, since the $\norm{\cdot}_2$ is generated by the inner product: \begin{equation*}
    f,g\in \ls{2}: \langle f,g \rangle := \int fg d\mu
\end{equation*}is well define by Holder’s inequality \ref{holder} $\int \abs{fg} d\mu \leq \norm{f}_2\norm{g}_2 < \infty$ \\
$\ls{2}$ is \textbf{Hilbert space}, i.e. it is a complete normed space with the norm generated by the inner product
\newpage 

\newpage
\begin{thm}[Dominated Convergence Theorem]
\label{DCT}
Let $f_n$ be sequence of measurable functions, if $\exists$ integrable $g$ s.t. $\abs{f_n} \leq g$ a.e. $\forall n \in \N$, then each $f_n$ is integrable. If $f_n \rightarrow f$ a.e. and $f$ is measurable, then $f$ is integrable
\begin{equation*}
    \biglim{n} \into \abs{f_n - f} d\mu = 0 \quad \text{and} \quad \biglim{n}\into f_n d\mu = \into f d\mu 
\end{equation*}
\end{thm}
\vspace{2cm}
\pf Since $\abs{f_n}\leq g$ a.e. Define $A_n := \{\omega| \abs{f_n}\geq g\}, B_n := \{\omega| \biglim{n}f_n \neq f\}$ \\
By definition we have $\mu(A_n)=\mu(B_n)=0\quad \forall n\in \N$ also,  $\mu(\bigu{n\geq1}(A_n\cup B_n)) = 0$ \\
Hence, on $(\bigu{n\geq1}(A_n\cup B_n))^c$ We have desired property that $\abs{f}\leq g$ a.e. \\
$\int \abs{f} d\mu \leq \int g d\mu < \infty$, $f$ is integrable and measurable, also we have $\abs{f_n - f} \leq \abs{f_n} +\abs{f} \leq 2g$ a.e. \\
Apply Fatou's Lemma we have:
\begin{align*}
    \int \linf{n}(2g - \abs{f_n-f}) d\mu &\leq \linf{n}\int 2g - \abs{f_n - f}d\mu \\
    &= \int 2g + \linf{n}\int -\abs{f_n-f} d\mu \\
    &= \int 2g - \lsup{n} \int \abs{f_n-f} d\mu \\
    LHS = \int \linf{n}(2g - \abs{f_n-f}) d\mu &= \int 2g \\
    \lsup{n} \int \abs{f_n-f} d\mu &\leq \int 2g - \int 2g = 0 \\
    0 \leq \linf{n}\int \abs{f_n-f} d\mu &\leq \lsup{n}\int \abs{f_n-f} d\mu \leq 0 \\
    \linf{n} = \lsup{n}, \text{ Limit exists and } &\biglim{n}\int \abs{f_n-f}) d\mu = 0 \\
    0 \leq \abs{\int (f_n-f)) d\mu} &\leq \int \abs{f_n-f}) d\mu \xrightarrow{n \rightarrow \infty} \\
    \biglim{n \rightarrow \infty} \int f_n d\mu &= \int f d\mu
\end{align*}
\newpage
\subsection{Modes of convergence}
\begin{dfn}
Sequence of measurable functions $f_n$ converges to a measurable function $f$:
\begin{enumerate}
    \item Almost everywhere
    \begin{equation*}
        \mu(\{\omega| \biglim{n} f_n(\omega) = f(\omega)\}^c) = 0
    \end{equation*}
    This is denoted by $f_n \xrightarrow{\text{a.e.}}f$
    \item In measure 
    \begin{equation*}
        \forall \epsilon >0 \quad \mu(\{\omega| \abs{f_n(\omega) - f(\omega)}>\epsilon \}) \xrightarrow{n \rightarrow \infty} 0
    \end{equation*}
    This is denoted by $f_n \xrightarrow{\mu}f$
    \item In $\ls{p} \quad (1\leq p \leq \infty)$ if $f_n\in \ls{p}$ \begin{equation*}
        \int \abs{f_n - f}^p d\mu \xrightarrow{n \rightarrow \infty} 0
    \end{equation*}
    This is denoted by $f_n \xrightarrow{\ls{p}}f$
    \item In distribution \\
    Let $(\mu_n), \mu$ be probability measures on $(\R, \B)$, we say $\mu_n$ converges weakly to $\mu$.\\
    If for any $x\in \R$ s.t. $\mu(\{x\}) = 0$ we have 
\begin{equation*}
\mu_n((-\infty,x]) \rightarrow \mu((-\infty,x]) 
\end{equation*}That is $\forall x\in \R $ s.t. F is continuous 
\begin{equation*}
    F_n(x) \xrightarrow{d} F
\end{equation*}This is denoted by $\mu_n \implies\mu$ or $\mu_n \xrightarrow{d}\mu$
\end{enumerate}












\end{dfn}
\begin{lem} if $f_n \xrightarrow{\text{a.e.}}f$ and $\mu(\Omega) < \infty$ then $f_n \xrightarrow{\mu}f$
\end{lem}
\begin{lem}\label{l1mu}
Let $f_n$ be measurable functions and $f_n \xrightarrow{\ls{1}}f$ Then $f_n \xrightarrow{\mu}f$
\end{lem}
\begin{lem}[Scheffe's lemma] Suppose $f_n, f\in \ls{1}, f_n \xrightarrow{\text{a.e.}}f$ Then
\begin{equation*}
    f_n \xrightarrow{\ls{1}}f \Longleftrightarrow \int \abs{f_n} d\mu \rightarrow \int \abs{f} d\mu
\end{equation*}
\end{lem}

\pf of Lemma \ref{l1mu} Fix $\epsilon >0, \abs{f_n - f}\in \F^+$, Then by Markov inequality \ref{markov}
\begin{equation*}
    \mu(\abs{f_n-f} \geq \epsilon) \leq \frac{1}{\epsilon}\int \abs{f_n-f} d\mu \xrightarrow{n\rightarrow \infty} 0
\end{equation*}
\begin{lem}
if $X_n \xrightarrow{\prob}$ then $X_n \xrightarrow{d} X$
\end{lem}
This proof is given at end of chapter 9
\newpage
\begin{example}Difference between modes of convergence.(Counterexample)
\begin{enumerate}
    \item $f_n \xrightarrow{\text{a.e.}}f \notimplies f_n \xrightarrow{\ls{1}}f$ \\
    Let $f_n := n \I_{(0,\frac{1}{n})}$ on $((0,1),\B, \lambda)$.
    \begin{align*}
        &f_n(\omega) \rightarrow 0 \quad \forall \omega \in \Omega \\
        &\int \abs{f_n - f} d\mu = \int f_n d\mu = n\lambda((1, \frac{1}{n})) = 1 \neq 0
    \end{align*}Note: we have $f_n \xrightarrow{\mu}f$ \\
    Let $\epsilon > 0\quad \lambda(\abs{f_n-0} > \epsilon) = \lambda(n\I_{(0,\frac{1}{n})} > \epsilon) \leq \frac{1}{n} \xrightarrow{n \rightarrow \infty}  0$
    \item $f_n \xrightarrow{\ls{1}}f + f_n \xrightarrow{\mu}f \notimplies f_n \xrightarrow{\text{a.e.}}f$ \\
    Define $f: f_1 = \I_{(0,\frac{1}{2})}, f_2 = \I_{(\frac{1}{2},1)}, f_3 = \I_{(0,\frac{1}{4})}, f_4 = \I_{(\frac{1}{4},\frac{1}{2})} ...$
    \begin{align*}
        &\int \abs{f_n - f} d\mu = \int f_n d\mu = \lambda(\text{support }f_n) = \xrightarrow{n \rightarrow \infty}  0 \\
        &\lambda(\abs{f_n} > \epsilon) \leq \lambda(\text{support }f_n) \xrightarrow{n \rightarrow \infty}  0 \\
        &\forall \omega\in\Omega, f_n(\omega) = (0,0,1,...,0,0,1,0,..)\qquad \exists\infty \text{ many n's the sequence does not converge for all omega}
    \end{align*}
    \item $f_n \xrightarrow{\text{a.e.}}f \notimplies f_n \xrightarrow{\mu}f$ \\
    Let $f_n := \I_{[n,\infty)}$ on $([0,\infty),\B, \lambda)$.
    \begin{align*}
       & \forall \omega\in [0,\infty), f_n \xrightarrow{\text{a.e.}} 0 \\
       &\lambda(\abs{f_n(\omega) \geq \frac{1}{2}}) = \lambda([n,\infty)) =\infty \neq 0 \\
    \end{align*}
\end{enumerate}
\end{example}
\newpage
\begin{thm}
Suppose that$f_n \xrightarrow{\mu}f$, Then $\exists$ a subsequences $\{n_k\}$ s.t. $f_{n_k} \xrightarrow{\text{a.e.}}f$	 
\end{thm}
\begin{rem}
Since $f_n \xrightarrow{\ls{1}}f \implies f_n \xrightarrow{\mu}f$, this claim also holds if we have $\ls{1}$ convergence.
\end{rem}
\pf For any $k\geq 1 \quad \exists n_k \geq 1$ s.t. 
\begin{equation*}
    \mu(\abs{f_n-f} > \frac{1}{2^k}) < \frac{1}{2^k} \qquad \forall n\geq n_k
\end{equation*}WLOG, we could assume $n_k \uparrow$ \\
Let $A_k := \{\omega| \abs{f_{n_k}(\omega) - f(\omega)}>\frac{1}{2^k}\} \qquad \bigs{k}\mu(A_k) \leq \bigs{k}\frac{1}{2^k} \leq \infty$ \\
by Borel-cantelli's Lemma \ref{BCL1} we have $\mu(\lsup{k}A_k) = 0$ \\
That is, $A_k$ occurs infinitely often have zero measure, that is for a.e. $\omega \in \Omega$, omega is inside of finitely many of $A_k's$ \\
For a.e $\omega \in \Omega \quad \exists K(\omega) \geq 1$ s.t $\omega \notin A_k \forall k\geq K(\omega)$ we have $\abs{f_{n_k}(\omega)-f(\omega)} \leq \frac{1}{2^k} \xrightarrow{k\rightarrow\infty} 0$
\qed
\begin{thm}
If $f_n \xrightarrow{\mu}f$ and $\abs{f_n}\leq g$ a.e with $g\in \ls{1}$ then $f_n \xrightarrow{\ls{1}}f$ (Stronger than DCT which requires a.e convergence for $\mu(\Omega) < \infty$)
\end{thm}
\pf Suppose $f_n \nrightarrow f$ in $\ls{1}$ Then $\exists \epsilon >0$ and $n_k \uparrow \infty $ s.t
\begin{equation*}
    \int \abs{f_{n_k}-f} d\mu > \epsilon, \qquad \forall k \qquad (*)
\end{equation*}But $f_{n_k} \xrightarrow{\mu}f$ By previous Thm, $\exists m_k\subset n_k$ s.t $f_{m_k} \xrightarrow{\text{a.e.}}f$ \\
By DCT we have $\biglim{n}\int \abs{f_{m_k}-f} = 0, f_{m_k} \xrightarrow{\ls{1}}f$ which contradicts (*)
\qed
\newpage
\subsection{Inequalities}
\begin{thm}[Markov inequality]\label{markov} For $f\in \F^+$ and $\alpha > 0 $
\begin{equation*}
    \mu\{\omega| f(\omega) \geq \alpha\} \leq \frac{1}{\alpha}\int fd\mu
\end{equation*}
\end{thm}

The next inequality will be used in proving Holder's inequality
\begin{thm}[Young's inequality] for $\omega, z > 0$
\begin{equation*}
    \omega z \leq \frac{1}{p} \omega^p + \frac{1}{q} z^q
\end{equation*}
\end{thm}
\begin{thm}[Holder's inequality]\label{holder} Let $f, g$ be measurable functions, $p,q \in (1, \infty)$ with $\frac{1}{q}+\frac{1}{p} = 1$ Then
\begin{equation*}
    \int \abs{fg} d\mu \leq \left(\int \abs{f}^p  d\mu\right)^{\frac{1}{p}}\left(\int \abs{g}^q  d\mu\right)^{\frac{1}{q}} = \norm{f}_p \norm{g}_q
\end{equation*}
\end{thm}
\begin{thm}[Minkowski's inequality(triangle inequality for p-norms)]For measurable functions $f,g$ and $p\geq 1$
\begin{equation*}
    \norm{f+g}_p \leq \norm{f}_p +\norm{g}_p
\end{equation*}
\end{thm}

\newpage
\textbf{proof of Markov inequality:} \\
\pf \begin{equation*}
    f\geq f \I_{\{f\geq \alpha\}} \geq \alpha\I_{\{f\geq \alpha\}} \geq 0
\end{equation*}Integrate both side also we have monotonicity
\begin{equation*}
    \int \alpha\I_{\{f\geq \alpha\}} = \alpha\mu\{\omega| f(\omega) \leq \alpha\} \leq \int fd\mu \implies \mu\{\omega| f(\omega) \leq \alpha\} \leq \frac{1}{\alpha}\int fd\mu
\end{equation*}\qed

\textbf{Proof of Minkowski's inequality:} \\
\pf WLOG $f,g\geq 0$
For $p=1$, $\abs{f+g} \leq \abs{f}+\abs{g}$, Apply monotonicity of Lebesgue measure, we have desired relation. \\
If $p > 1$, let $q\in (1,+\infty)$ s.t. $\frac{1}{q} = 1- \frac{1}{p} \implies q(p-1) = p$ \\
we will utilise 
\begin{equation}\label{eq:1}
    (f+g)^p = f(f+g)^{p-1} + g(f+g)^{p-1}
\end{equation}
Apply Holder's inequality on RHS of above equation
\begin{align*}
    \int f(f+g)^{p-1} d\mu &\leq \left(\int \abs{f}^p  d\mu\right)^{\frac{1}{p}} \left(\int \abs{f+g}^{(p-1)q} d\mu\right)^{{\frac{1}{p}}{\frac{p}{q}}} = \norm{f}_p\norm{f+g}_p^{\frac{p}{q}} \\
    \int g(f+g)^{p-1} d\mu &\leq \norm{g}_p\norm{f+g}_p^{\frac{p}{q}}
\end{align*}
Integrate equation \ref{eq:1} we have 
\begin{align*}
    \int(f+g)^p d\mu &\leq \int f(f+g)^{p-1} d\mu + \int g(f+g)^{p-1} d\mu \\
    &= \norm{f}_p\norm{f+g}_p^{\frac{p}{q}} +\norm{g}_p\norm{f+g}_p^{\frac{p}{q}} \\
    &= \norm{f+g}_p^{\frac{p}{q}}(\norm{f}_p+\norm{g}_p)
\end{align*}LHS = $\int(f+g)^p d\mu = \norm{f+g}_p^p$ divided by $\norm{f+g}_p^{\frac{p}{q}}$ both sides we have
\begin{equation*}
    \norm{f+g}_p^{p(1-\frac{1}{q})} = \norm{f+g} \leq \norm{f}_p+\norm{g}_p
\end{equation*}
\qed